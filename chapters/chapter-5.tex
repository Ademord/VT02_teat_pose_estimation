
\chapter{Conclusion}\label{chap:conclusion}
% \section{Discussion and Conclusion}
% 3 pages
\section{Conclusion}]
In this work we could show that is it possible to acquire a deep scene understanding from sequential
data with supervised learning. In a simplified use-case such as the presented one, the system can
successfully acquire a spatial scene understanding that includes objects, their shape, color and even
their relationship with other objects. Although it is limited to the type of scenes the system was trained
on, its versatility exceeds by far current mainstream object recognition systems such as ResNet [6].
Unlike encoder-decoder-based scene understanding approaches such as [32], we do not need specified
3D models for training but only a simple scene description that specifies present objects including their
positions and the camera poses of the captured images. 

We could show that our system is capable of
sequentially integrating information from new frames into the existing scene embedding vector. This
capability is indeed highly remarkable because it includes the achievement of multiple non-trivial steps:
1.The change in camera position relative to the previous frame needs to be evaluated. The system does
not receive any information about the camera position or rotation in space, but needs to extract this
information based on the difference between the already perceived and the new input frame.2.This
extracted transformation of the camera position must be used to transform all remembered object
positions. Some objects might be occluded by the obstacle or other present objects, and the system is
not able to perceive these items in the current frame. Nonetheless, it is necessary to transform their
remembered positions so that the system is able to return the correct position when queried. The
precision of this process however, would need to be further improved when used for robotic grasping,
as there is still an average deviation of 0.11 meters (about half the diameter of the object) from the
ground truth when using the best performing approach.

While our proposed system does not use separated streams for ventral and dorsal pathways, our information
aggregation process is inspired by the quicker decaying dorsal memory and the more persistent
ventral memory. This is represented by learned weights versus aggregated temporary information. This
architecture seems to work great, especially with respect to the 3 different shortcomings that were the
focus of this work (see section 1.2). In the following, we discuss how our system overcomes them:
    % \begin{itemize}
    %     \item 1 paragraph big: describe what was achieved
    %     \item 1 paragraph: describe our limitations
    %     \item describe how we "overcame" the shortcommings: 3 paragraphs
    %     \item closing remarks
    % \end{itemize}
\section{Towards Active Vision}
Since this work was initially inspired by robotic interaction, we would see it as the next step to combine
our vision-focused system with grasping approaches. An interesting direction would be to work towards
solving benchmarks for robotic interaction such as RLBench [59].

When approaching such tasks, we mainly see two possible paths to take: One relatively simple way
would be to keep the perception and the grasping system separate and only use the output of the
presented approach as input for a grasp generation algorithm, such as Dex-Net [60]. This would mean
that the perception part would identify the target object and then forward its position to the grasp
planning mechanism which would plan the grasp and pass it to the robotic arm for execution. As
second option it would be possible that the condensed scene representation produced by our system
would be beneficial for grasp-generation. We think that a promising approach would be to train two
streams for grasping. The first stream could create a large number of possible grasps, while the other
one would rate them with likelihood of success. Of course, the system would first need to learn to
include the required information in the scene representation, which might be a large leap compared to
the information currently present within the trained system. However, with the human mind closely
coupling perception and action as part of the dorsal pathway it seems that such a joint approach could
work for robotic grasping too.
    % \begin{itemize}
    %     \item 1 big paragraph describing how active vision could improve accuracy and object permanence
    % \end{itemize}
\section{Where to Go From Here?}
While the discussed approach successfully solves the tasks set, the system is still far from being a real
replacement for existing computer vision algorithms in use. On the way to the application of a system
like ours to solve tasks in the real world, it would be necessary to solve at least some of the following
challenges:

Real world data: In order to use an approach like the one presented in a real-world use-case, it would
above all be necessary to transfer the approach to real data or at least more realistic synthetic data.
With the goal to jointly improve both scene understanding and active vision, we would inspire future
research to use real-time data gathering with simulated environments as for example Isaac Sim [61].
This would allow non-discrete view-positions and thus a potentially better understanding. One aspect
that still might not be solvable by using synthetic data is the noise of the depth-channel of the RGB-D
data, which is quite prominent for most consumer class RGB-D-camera.

More different object classes/shapes: The presented solution uses only 5 primitive shapes with 7
colors, which most likely does not reflect the conditions of a real world use case. A possible solution
could be the use of large-scale 3D object data sets as used for Dex-Net [60]. Closely related to more
diverse objects would be the capability to deal with duplicate objects. This could possibly be solved by
adding multi-object output to all streams (as demonstrated with the Enumerating stream).

Motion: In this work, we did not address the topic of motion, which does play a big roll in the real
world. Environments like conveyor belts would be a domain where a scene understanding system like
the one presented could prove highly useful. However, an application in such an environment would
require previous research with dynamic scenes.

Rotation axis detection: With only primitive objects such as cube, cylinders, etc., we decided not to
include rotation information for our streams. However, for a large number of use cases it could be very
useful to obtain the rotation of an object, so we would encourage future work to extend our approach
to include rotation information.
Grasping/bounding-box detection: For an actual application of robotic grasping, it would be necessary
to generate possible grasping positions. We did not address this topic in the course of this work
but leave the extension of the presented approach with grasping to further research. For more details
see section 5.3.
    % \begin{itemize}
    %     \item describe in 5 small paragraphs 5 different future improvements for the voerall system
    % \end{itemize}


